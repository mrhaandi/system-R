\label{sec:mset_dim}
Comparing the dimensionally bounded system \textbf{R} to the currently most expressive dimensionally bounded system $(\Vdash_k)$ with decidable inhabitation~\cite[Section~3.2]{DudenhefnerR17}, one can observe several differences.

The system $(\Vdash_k)$ introduces non-idempotent intersection types (multisets to the left of the arrow) as annotations throughout the derivation in order to capture intersection introduction as a resource.
%different from traditional literature~\cite{BucciarelliKV17}}
This induces additional bookkeeping (see side condition $(\star)$~\cite[Section~3.2]{DudenhefnerR17} in rule $(\cap\textnormal{I})$), and the  correspondence to established intersection type systems (both idempotent and non-idempotent) is not immediate.

The most important difference is that the individual dimensions in system $(\Vdash_k)$ are strictly less comprehensive with respect to inhabited terms.
First, any derivation in $(\Vdash_k)$ can be translated to a derivation in $(\vdash_k)$.

\begin{lemma}\label{lem:multiset_to_vector}
    If $\Gamma_{[1]} \Vdash_k t : A$ and $\dim(\Gamma) = 1$, then $\Gamma \vdash_k t : (A)$.
\end{lemma}

\begin{proof}
    Both ($\Vdash_k$) and ($\vdash_k$) follow derivations in ($\vdash_S$).
    Using lexicographic ordering, multisets of cardinality at most $k$ at top-level can be captured by vectors of size at most $k$.
    Therefore, the same construction as the proof of Lemma~\ref{lemma:fromvanBakel} applies for the translation.    
\end{proof}

Since inhabitation in $(\Vdash_k)$ is \textsf{EXPSPACE}-complete~\cite[Theorem~34]{DudenhefnerR17}, the above Lemma~\ref{lem:multiset_to_vector} implies \textsf{EXPSPACE}-hardness of inhabitation in $(\vdash_k)$.

\begin{corollary}\label{cor:inh_lower_bound}
    Given $\Gamma$, $\bar{A}$, and $k$ it is \textsf{EXPSPACE}-hard to decide whether the judgment $\Gamma \vdash_k t : \bar{A}$ is derivable for some term $t$.
\end{corollary}

Second, the following Lemma~\ref{lem:exp_blowup_1} shows that there are types $A_1, A_2, \ldots$ which are inhabited in dimension $2$, but each $A_i$ requires multiset dimension $2^i$ to be assigned to any term. 

\begin{lemma}\label{lem:exp_blowup_1}
    There exists a family of types $\{A_1, A_2, \ldots\}$ such that for all $n \in \mathbb{N}$:
    \begin{enumerate}
        \item The size (number of nodes in the syntax tree) of $A_n$ is linear in $n$.
        \item There exists a term $t_n$ such that $\emptyset \vdash_2 t_n : (A_n)$.
        \item If there exists a term $u_n$ such that $\emptyset \Vdash_k u_n : A_n$, then $k \geq 2^n$.
    \end{enumerate}
\end{lemma}

\begin{proof}
    Let $\sigma_n := \bigcup\limits_{i=1}^n \{ \{a_i, b_i\} \to a_{i+1}, \{a_i, b_i\} \to b_{i+1} \}$,
    $\tau_n := \{\{a_{n+1}, b_{n+1}\} \to c\}$
    and $A_n := \{a_0, b_0\} \to \sigma_n \to \tau_n \to c$.
    \begin{enumerate}
        \item This property holds by definition.
        \item Let $v_0 := x$ and $v_{i+1} := y\,v_i$ for $i \in \{0, \ldots, n\}$ and let $t_n := \lambda xyz.z\,v_n$.
        We have $\emptyset \vdash_2 t_n : (A_n)$.
        The crucial iterated step for $i = 1, \ldots n$ is
            \begin{displaymath}
       {\RightLabel{\textnormal{($\Rightarrow$E)}}
         \AxiomC{$\Gamma \vdash_2 y : R((a_i, b_i)) \Rightarrow (a_{i+1}, b_{i+1})$}
        \AxiomC{$\Gamma \vdash_2 v_i : (a_i, b_i)$}
        \BinaryInfC{$\Gamma \vdash_2 y\,v_i : (a_{i+1}, b_{i+1})$}
        \DisplayProof
  }
    \end{displaymath}
    Where for the environment $\Gamma := \{x: (\{a_0, b_0\}, \{a_0, b_0\}), y: (\sigma_n, \sigma_n), {z: (\tau_n, \tau_n)} \}$ and the relation $R := \{(1,1), (1,2), (2, 1), (2,2)\}$ we observe that $R(\Gamma) = \Gamma$.
    \item An inhabitant of $A_n$ is necessarily of shape $\lambda xyz.z\,(y \ldots (y\,x) \ldots)$ such that $y$ occurs at least $n$ times and each occurrence of $y$ doubles the annotation multiset cardinality.\qedhere
    \end{enumerate}
    
\end{proof}

Third, complementarily to the above lemma, there is a family of terms which are typed at a specific type $A$ in dimension $2$, but require an exponentially large multiset dimension.

\begin{lemma}\label{lem:exp_blowup_2}
    There exists a type $A$ and a family of terms $\{t_1, t_2, \ldots\}$ such that for all $n \in \mathbb{N}$:
    \begin{enumerate}
        \item The size (number of nodes in the syntax tree) of $t_n$ is linear in $n$.
        \item $\emptyset \vdash_2 t_n : (A)$.
        \item If $\emptyset \Vdash_k t_n : A$, then $k \geq 2^n$.
    \end{enumerate}
\end{lemma}

\begin{proof}
    Let $u_0 := x$ and $u_{n+1} := y\,u_n$ for $n \in \mathbb{N}$, and let $t_n := \lambda xyz.z\,u_n$.
    Additionally, let $\sigma := \{\{a, b\} \to a, \{a, b\} \to b\}$,
    let $\tau := \{\{a, b\} \to c\}$,
    and let $A := \{a, b\} \to \sigma \to \tau \to c$.
    \begin{enumerate}
        \item This property holds by definition.
        \item We have $\emptyset \vdash_2 t_n : (A)$.
        The crucial step for $i \in \mathbb{N}$ is
            {\begin{displaymath}
            \RightLabel{\textnormal{($\Rightarrow$E)}}
            \AxiomC{$\Gamma \vdash_2 y : R((a, b)) \Rightarrow (a, b)$}
            \AxiomC{$\Gamma \vdash_2 u_i : (a, b)$}
            \BinaryInfC{$\Gamma \vdash_2 y\,u_i : (a, b)$}
            \DisplayProof
            \end{displaymath}}%
        Where $\Gamma := \{x: (\{a, b\}, \{a, b\}), y: (\sigma, \sigma), z: (\tau, \tau) \}$
    and $R := \{(1,1), (1,2), (2, 1), (2,2)\}$, observing that $R(\Gamma) = \Gamma$.
        \item Similarly to the proof of Lemma~\ref{lem:exp_blowup_1}, each occurrence of $y$ via rule $(\cap\textnormal{I})$  doubles the annotation multiset cardinality.\qedhere
    \end{enumerate}
\end{proof}

The avid reader may wonder: why is the \textsf{EXPSPACE} inhabitation procedure~$\mathcal{A}_{\langle d \rangle}$~\cite{DudenhefnerR17}, which respects a bound $d$ on the multiset dimension, not suited for inhabitation in dimensionally bounded system~\textbf{R}?
One immediate answer is given by the construction in the proof of Lemma~\ref{lem:exp_blowup_1}.
Iterative intersection introduction strictly increases multiset dimension.
This is reflected in procedure~$\mathcal{A}_{\langle d \rangle}$ via \emph{multisets of simultaneous constraints} with strictly increasing cardinalities on intersection introduction.

Still, the immediate answer is not satisfactory.
The construction in the proof of Lemma~\ref{lem:exp_blowup_1} can easily be accommodated for by identifying redundant constraints.
The modified inhabitation procedure would remain in \textsf{EXPSPACE}.

A key property of procedure~$\mathcal{A}_{\langle d \rangle}$ is that environments in simultaneous constraints are monotone with respect to pointwise inclusion.
That is, there is no need to strengthen type assumptions.
This is different for the system~\textbf{R} for which in rule ($\Rightarrow$E) type assumptions of premises are combined, including the action of a relation $R$.
This violates the monotonicity condition, and the \textsf{EXPSPACE} (or, alternating exponential time) argument is not applicable.

Overall, the present notion of dimension lies strictly between the multiset dimension (\textsf{EXPSPACE}-complete inhabitation~\cite[Theorem~34]{DudenhefnerR17}) and set dimension (undecidable inhabitation~\cite[Theorem~28]{DudenhefnerR17}), while preserving decidability of inhabitation.
