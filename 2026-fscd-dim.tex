\documentclass[a4paper,USenglish,cleveref, autoref, thm-restate]{lipics-v2021}
\usepackage[utf8]{inputenc}
\usepackage[T1]{fontenc}
\usepackage{amsmath}
\usepackage{amsfonts}
\usepackage{amssymb}
\usepackage{amsthm}
\usepackage{bussproofs}
\usepackage{graphicx}
\usepackage{lmodern}
\usepackage{microtype}
\usepackage{csquotes}
\usepackage{xcolor}
\usepackage{hyperref}
\usepackage{stmaryrd}
\usepackage{xspace}
%This is a template for producing LIPIcs articles. 
%See lipics-v2021-authors-guidelines.pdf for further information.
%for A4 paper format use option "a4paper", for US-letter use option "letterpaper"
%for british hyphenation rules use option "UKenglish", for american hyphenation rules use option "USenglish"
%for section-numbered lemmas etc., use "numberwithinsect"
%for enabling cleveref support, use "cleveref"
%for enabling autoref support, use "autoref"
%for anonymizing the authors (e.g. for double-blind review), add "anonymous"
%for enabling thm-restate support, use "thm-restate"
%for enabling a two-column layout for the author/affilation part (only applicable for > 6 authors), use "authorcolumns"
%for producing a PDF according the PDF/A standard, add "pdfa"

%\pdfoutput=1 %uncomment to ensure pdflatex processing (mandatatory e.g. to submit to arXiv)
%\hideLIPIcs  %uncomment to remove references to LIPIcs series (logo, DOI, ...), e.g. when preparing a pre-final version to be uploaded to arXiv or another public repository

%\graphicspath{{./graphics/}}%helpful if your graphic files are in another directory

\bibliographystyle{plainurl}% the mandatory bibstyle

\title{Bounded Parallel Intersection Type System} %TODO Please add

%\titlerunning{Dummy short title} %TODO optional, please use if title is longer than one line

\author{Andrej Dudenhefner}{TU Dortmund University, Germany}{andrej.dudenhefner@cs.tu-dortmund.de}{https://orcid.org/0000-0003-1104-444X}{}{}%TODO mandatory, please use full name; only 1 author per \author macro; first two parameters are mandatory, other parameters can be empty. Please provide at least the name of the affiliation and the country. The full address is optional. Use additional curly braces to indicate the correct name splitting when the last name consists of multiple name parts.

\author{Aleksy Schubert}{University of Warsaw, Poland}{alx@mimuw.edu.pl}{}{}{}

\author{Jakob Rehof}{TU Dortmund University, Germany}{jakob.rehof@cs.tu-dortmund.de}{}{}{}

\authorrunning{A. Dudenhefner, A. Schubert, and J. Rehof} %TODO mandatory. First: Use abbreviated first/middle names. Second (only in severe cases): Use first author plus 'et al.'

\Copyright{Andrej Dudenhefner, Aleksy Schubert, and Jakob Rehof} %TODO mandatory, please use full first names. LIPIcs license is "CC-BY";  http://creativecommons.org/licenses/by/3.0/

\ccsdesc[100]{\textcolor{red}{Replace ccsdesc macro with valid one}} %TODO mandatory: Please choose ACM 2012 classifications from https://dl.acm.org/ccs/ccs_flat.cfm 

\keywords{Dummy keyword} %TODO mandatory; please add comma-separated list of keywords

\category{} %optional, e.g. invited paper

%\relatedversion{} %optional, e.g. full version hosted on arXiv, HAL, or other respository/website
%\relatedversiondetails[linktext={opt. text shown instead of the URL}, cite=DBLP:books/mk/GrayR93]{Classification (e.g. Full Version, Extended Version, Previous Version}{URL to related version} %linktext and cite are optional

%\supplement{}%optional, e.g. related research data, source code, ... hosted on a repository like zenodo, figshare, GitHub, ...
%\supplementdetails[linktext={opt. text shown instead of the URL}, cite=DBLP:books/mk/GrayR93, subcategory={Description, Subcategory}, swhid={Software Heritage Identifier}]{General Classification (e.g. Software, Dataset, Model, ...)}{URL to related version} %linktext, cite, and subcategory are optional

%\funding{(Optional) general funding statement \dots}%optional, to capture a funding statement, which applies to all authors. Please enter author specific funding statements as fifth argument of the \author macro.

%\acknowledgements{I want to thank \dots}%optional

%\nolinenumbers %uncomment to disable line numbering



%Editor-only macros:: begin (do not touch as author)%%%%%%%%%%%%%%%%%%%%%%%%%%%%%%%%%%
\EventEditors{John Q. Open and Joan R. Access}
\EventNoEds{2}
\EventLongTitle{42nd Conference on Very Important Topics (CVIT 2016)}
\EventShortTitle{CVIT 2016}
\EventAcronym{CVIT}
\EventYear{2016}
\EventDate{December 24--27, 2016}
\EventLocation{Little Whinging, United Kingdom}
\EventLogo{}
\SeriesVolume{42}
\ArticleNo{23}
%%%%%%%%%%%%%%%%%%%%%%%%%%%%%%%%%%%%%%%%%%%%%%%%%%%%%%


\DeclareMathOperator{\dom}{dom}
\DeclareMathOperator{\var}{var}
\DeclareMathOperator{\rank}{rank}
\DeclareMathOperator{\form}{form}

\newcommand{\rbar}[1]{\mathrel{\bar{#1}}}
\newcommand{\edmark}{{\color{red} mark} }

\begin{document}

\maketitle

%TODO mandatory: add short abstract of the document
\begin{abstract}
Bounded Parallel Intersection Type System
\end{abstract}

\section{Bounded Parallel Intersection Type System}

We denote \emph{intersection types} by $A, B$, finite sets of intersection types by $\sigma, \tau$, and \emph{type atoms} by $a, b$.

\begin{definition}[Intersection Types]
\label{def:itypes} ~

$\begin{array}{rcl}
%\mbox{Types} \ni & 
A, B &::=& a \mid \sigma \to A\\
%\{\}\mbox{-Types}\ni &
\sigma, \tau &::=& \{A_1, \ldots, A_n\} \text{ where } n \geq 0
\end{array}$
\end{definition}
Intersection types are a variant of strict types considered by van Bakel~\cite{Bakel92,Bakel04,Bakel11}.

We denote vectors of types and sets of types by $\bar{A}$ and $\bar{\sigma}$ respectively.
If $\bar{\sigma} = (\sigma_1, \ldots, \sigma_n)$ and $\bar{A} = (A_1, \ldots, A_n)$, then $\bar{\sigma} \Rightarrow \bar{A} = (\sigma_1 \to A_1, \ldots, \sigma_n \to A_n)$, in particular $()\Rightarrow()=()$. We write $\bar{A}_{[i]}$ for $A_i$ and $\bar{\sigma}_{[i]}$ for $\sigma_i$.
The \emph{dimension} of a vector is the number of its \emph{coordinates}, that is $\dim(A_1, \ldots, A_n) = n$.
We extend relations $\subseteq$, $\in$, and the operation $\cup$ to vectors.
\begin{definition}[Vector Extension]~
    \begin{itemize}
        \item $(\sigma_1, \ldots, \sigma_n) \rbar{\subseteq} (\tau_1, \ldots, \tau_n)$ if $\sigma_1 \subseteq \tau_1, \ldots, \sigma_n \subseteq \tau_n$
        \item $(A_1, \ldots, A_n) \rbar{\in} (\sigma_1, \ldots, \sigma_n)$ if $A_1 \in \sigma_1, \ldots, A_n \in \sigma_n$
        \item $(\sigma_1, \ldots, \sigma_n) \rbar{\cup} (\tau_1, \ldots, \tau_n) := (\sigma_1 \cup \tau_1, \ldots, \sigma_n \cup \tau_n)$
        \item $(\{A_1\}, \ldots, \{A_n\})\!\!\downarrow\, := (A_1, \ldots, A_n)$
    \end{itemize}
\end{definition}

\begin{definition}[Environment of Dimension $d$]
An \emph{environment} $\Sigma$ is a finite set of \emph{type assumptions} having the shape $x : \sigma$ for distinct term variables. An \emph{environment} $\Gamma$ of dimension $d$ is a finite set of \emph{type assumptions} having the shape $x : \bar{\sigma}$ for distinct term variables such that $\dim(\bar{\sigma}) = d$.
%$\Gamma ::=\{x_1 : \bar{\sigma_1}, \ldots, x_n : \bar{\sigma_n}\}$ where $x_i \neq x_j$ for $i \neq %j$ and $\dim(\Gamma) := \dim(\bar{\sigma_1}) = \cdots = \dim(\bar{\sigma_n}) = d$.
\end{definition}
Each environment $\Gamma$ is equipped with its dimension $\dim(\Gamma)$.
In particular, the empty environment of dimension $0$ is distinct from the empty environment of dimension $1$. 

We may extend an environment $\Gamma$ by an additional assumption $x : \bar{\sigma}$, written $\Gamma, x : \bar{\sigma}$, where $\dim(\bar{\sigma}) = \dim(\Gamma)$ and $x$ does not appear in any assumption in $\Gamma$.
The \emph{domain} of an environment $\Gamma$ is given by $\dom(\Gamma) = \{ x \mid (x : \bar{\sigma}) \in \Gamma \text{ and } \bar{\sigma} \neq   (\emptyset, \ldots, \emptyset)\}$.
When $(x:\bar{\sigma})\in\Gamma$ we write $\Gamma(x)$ for $\bar{\sigma}$.
We identify $\Gamma$ and $\Gamma, x : (\emptyset, \ldots, \emptyset)$, that is $\Gamma(y) = (\emptyset, \ldots, \emptyset)$ iff $y \not\in \dom(\Gamma)$ for any term variable $y$.
We write $\Gamma_{[i]}$ for $\{ x:\bar{\sigma}_{[i]} \mid x:\bar{\sigma}\in\Gamma\}$.
We extend the relation $\rbar{\subseteq}$ to environments of the same dimension so that $\Gamma \rbar{\subseteq} \Delta$ when for each $x \in \dom(\Gamma)$ we have $\Gamma(x) \rbar{\subseteq} \Delta(x)$.
If $\dim(\Gamma_1) = \dim(\Gamma_2)$, we define $\Gamma_1\cup\Gamma_2$ {\color{red} should we write $\Gamma_1 \rbar{\cup} \Gamma_2$ for consistency?} such that $(\Gamma_1\cup\Gamma_2)(x) = \Gamma_1(x) \rbar{\cup} \Gamma_2(x)$.

\begin{lemma}[Properties of Type Environments]
\label{lem:env-properties}~
\begin{enumerate}
\item If $\Gamma\rbar\subseteq\Delta$ then $\Gamma\cup\Delta=\Delta$.
\end{enumerate}
\end{lemma}

Extending notation~\cite{DudenhefnerU21} from surjective functions to binary relations we define type vector transformations as follows.

\begin{definition}[Type Vector Transformation]
Let $R \subseteq \{1, \ldots, n\} \times \{1, \ldots, m\}$ be a binary relation where $n, m \geq 0$.
\begin{itemize}
\item $R(A_1, \ldots, A_n) := (\tau_1, \ldots, \tau_m)$ where $\tau_i = \{A_j \mid j \, R\, i\}$
\item $R(\sigma_1, \ldots, \sigma_n) := (\tau_1, \ldots, \tau_m)$ where $\tau_i = \bigcup \{\sigma_j \mid j \, R\, i\}$
\item $R(\{x_1 : \bar{\sigma_1}, \ldots, x_l : \bar{\sigma_l}\}) := \{x_1 : R(\bar{\sigma_1}), \ldots, x_l : R(\bar{\sigma_l})\}$
\end{itemize}
\end{definition}

\begin{example}
For $R = \{(1, 1), (1, 3), (2, 1), (3, 2), (3, 3)\}$ we have \[R((a, b, c)) = (\{a, b\}, \{c\}, \{a, c\})\]
\end{example}

\begin{definition}
Let $R\subseteq \{1, \ldots, n\} \times \{1, \ldots, m\}$  be a relation.
\begin{itemize}
    \item $i\,R\,j$ means $(i,j)\in R$
    \item $R$ is \emph{left-total} when for each $i\in \{1, \ldots, n\}$ there is a $j$ such that $i\,R\,j$
    \item $R$ is \emph{right-total} when for each $j\in \{1, \ldots, m\}$ there is an $i$ such that $i\,R\,j$
    \item $R$ is \emph{left-unique} when for each $i\in\{1, \ldots, n\}$ there is at most one $j$ such that $i\,R\,j$
    \item $R$ is \emph{right-unique} when for each $j\in\{1, \ldots, m\}$ there is at most one $i$ such that $i\,R\,j$
\end{itemize}
\end{definition}

If a relation $R$ is right-unique ({\color{red} you also need right-totality, otherwise you get some empty sets}) we have $R(\bar{B}) = (\{A_1\},\ldots, \{A_m\})$, and therefore $R(\bar{B})\!\!\downarrow\, = (A_1,\ldots,A_m)$ is well-defined.

Preliminary properties of vector transformations are listed in the following Lemma~\ref{lem:rel-properties}.

\begin{lemma}[Properties of Vector Transformations]
\label{lem:rel-properties}~
\begin{enumerate}
    \item For two binary relations $R, R' \subseteq \{1, \ldots, n\} \times \{1, \ldots, m\}$ we have that if $R \subseteq R'$, then for all $\bar{\sigma}$ such that $\dim(\bar{\sigma}) = n$ we have $R(\bar{\sigma}) \rbar{\subseteq} R'(\bar{\sigma})$.
    \item If $R(\bar{\sigma}) \rbar{\subseteq} R'(\bar{A})$ and $R$ is left-total, then there exists $R''$ such that $R''(\bar{A}) = \bar{\sigma}$.
    \item If $\bar{A}$ has distinct coordinates, $R''(\bar{A}) = \bar{\sigma}$, $R(\bar{\sigma}) \rbar{\subseteq} R'(\bar{A})$, and $R$ is left-total, then $R \circ R'' \subseteq R'$.
    \item $R(\Gamma_1 \cup \Gamma_2) = R(\Gamma_1) \cup R(\Gamma_2)$
    \item If $R$ is right-unique and right-total, and $\bar{A} \rbar{\in} \bar{\sigma}$ then
        $R(\bar{A})\!\!\downarrow \rbar{\in} R(\bar{\sigma})$.
    \item If $R$ is right-unique and right-total, then $R(\bar{\sigma} \Rightarrow \bar{A})\!\!\downarrow = R(\bar{\sigma}) \Rightarrow R(\bar{A})\!\!\downarrow$.
    \item\label{prop:rel-properties:saturate} If $R(\bar{\sigma}) \rbar{\subseteq} \bar{\tau}$ and $R$ is left-total, then there exists $\bar{\sigma}'$ such that $\bar{\sigma} \rbar{\subseteq} \bar{\sigma}'$, $R(\bar{\sigma}') \rbar{\subseteq} \bar{\tau}$, and for all $i \in \{1, \ldots, \dim(\bar{\sigma}')\}$ there exists $j \in \{1, \ldots, \dim(\bar{\tau})\}$ such that $\bar{\sigma}'_{[i]} = \bar{\tau}_{[j]}$.
\end{enumerate}
\end{lemma}

\begin{proof}~
    \begin{enumerate}
        \item The claim follows from $R(\bar{\sigma})_{[i]} = \bigcup \{\sigma_{[j]} \mid j \, R\, i\} \subseteq \bigcup \{\sigma_{[j]} \mid j \, R'\, i\} = R'(\bar{\sigma})_{[i]}$.
        \item  Consider $R'' = \{ (i, j) \mid \exists k.i\, R'\, k \mbox{ and } j\, R\, k \mbox{ and } \bar{A}_{[i]}\in\bar\sigma_{[j]}\}$.
            We have $R''(\bar{A})_{[j]} = \{ \bar{A}_{[i]} \mid i\, R''\, j \} = \{ \bar{A}_{[i]} \mid  \exists k.i\, R'\, k \mbox{ and } j\, R\, k \mbox{ and } \bar{A}_{[i]}\in\bar\sigma_{[j]}\}$.
            This set is included in $\bar\sigma_{[j]}$ as for each its element $\bar{A}_{[i]}$ we have
            $\bar{A}_{[i]}\in\bar\sigma_{[j]}$. Similarly, $\bar\sigma_{[j]}\subseteq R(\bar\sigma)_{[k]}$ for some $k$ as $R$ is left-total. As $R(\bar\sigma)_{[k]}\subseteq R'(\bar{A})_{[k]}$, each element of $\bar{\sigma}_{[j]}$ must be equal to some $\bar{A}_{[i]}$ such that $i\, R'\, k$.
        \item We have $(R \circ R'')(\bar{A}) = R(\bar{\sigma}) \rbar{\subseteq} R'(\bar{A})$.
        Since $\bar{A}$ has distinct coordinates, we obtain the claim.     
        \item Immediate from the definition.
        \item Since $R$ is right-unique and right-total, for each $i \in \{1, \ldots, \dim(R(\bar{A})\!\!\downarrow)\}$ there is exactly one $j$ such that $j\,R\,i$, and we have  $(R(\bar{A})\!\!\downarrow)_{[i]} = \bar{A}_{[j]} \in \bar{\sigma}_{[j]} \subseteq R(\bar{\sigma})_{[i]}$.
        \item Since $R$ is right-unique and right-total, we have $(R(\bar{\sigma} \Rightarrow \bar{A})\!\!\downarrow)_{[i]}\, = \bar{\sigma}_{[j]}\to\bar{A}_{[j]}$ where $j\, R\, i$ for exactly one $j$. 
        Similarly, $R(\bar{\sigma})_{[i]} =  \bar{\sigma}_{[j]}$ and %
        $(R(\bar{A})\!\!\downarrow)_{[i]} = \bar{A}_{[j]}$ where $j\, R\, i$. Consequently, we obtain $R(\bar{\sigma} \Rightarrow \bar{A})\!\!\downarrow = R(\bar{\sigma}) \Rightarrow R(\bar{A})\!\!\downarrow$.
        \item Since $R$ is left-total, for all $i \in \{1, \ldots, \dim(\bar{\sigma})\}$ there exists $j \in \{1, \ldots, \dim(\bar{\tau})\}$ such that $i\,R\,j$, for which we set $\bar{\sigma}'_{[i]} := \bar{\tau}_{[j]}$, satisfying the required properties.
    \end{enumerate}
\end{proof}

Judgments in the following \emph{parallel intersection type system} with a \emph{dimension bound} $k$ are of shape $\Gamma \vdash_k t : \bar{A}$ such that $\dim(\Gamma) = \dim(\bar{A})$.

\begin{definition}[Parallel Intersection Type System of Bounded Dimension]
\label{def:bounded-type-system}
~\\

\begin{minipage}{\textwidth}
\centering
\begin{tabular}{l}
{\RightLabel{\textnormal{(Ax)}}
\AxiomC{$A_i \in \sigma_i$ for $i = 1, \ldots, n$}
\AxiomC{$n \leq k$}
\BinaryInfC{$\Gamma, x : (\sigma_1, \ldots, \sigma_n) \vdash_k x : (A_1, \ldots, A_n)$}
\DisplayProof}
\\\\
{\RightLabel{\textnormal{($\Rightarrow$I)}}
\AxiomC{$\Gamma, x: \bar{\sigma} \vdash_k t : \bar{A}$}
\UnaryInfC{$\Gamma \vdash_k \lambda x.t : \bar{\sigma} \Rightarrow \bar{A}$}
\DisplayProof}\quad
{\RightLabel{\textnormal{($\Rightarrow$E)}}
\AxiomC{$\Gamma \vdash_{k} t : R(\bar{A}) \Rightarrow \bar{B}$}
\AxiomC{$\Delta \vdash_{k} u : \bar{A}$}
\BinaryInfC{$\Gamma \cup R(\Delta) \vdash_k t \; u : \bar{B}$}
\DisplayProof}
\end{tabular}
\end{minipage}
\end{definition}

We may leave out the actual bound, if it is immaterial.

\begin{definition}[Parallel Intersection Type System]
    $\Gamma \vdash t : \bar{A}$ means $\Gamma \vdash_k t : \bar{A}$ for some $k$.
\end{definition}

\begin{lemma}
The following rule is derivable:
{\RightLabel{\textnormal{($nil$)}}
\AxiomC{\phantom{1337}}
\UnaryInfC{$\emptyset \vdash_k t : ()$}
\DisplayProof}.
\end{lemma}

\begin{proof}
We proceed by induction on $t$.
\begin{description}
    \item[Case $t = x$:] Due to $\emptyset, x : () = \emptyset$, by rule (Ax) we have $\emptyset \vdash_k x: ()$.
    \item[Case $t = \lambda x.u$:] By the induction hypothesis we have $\emptyset \vdash_k u : ()$.
    Due to $\emptyset, x : () = \emptyset$ and $() \Rightarrow () = ()$, by rule ($\Rightarrow$I) we have $\emptyset \vdash_k \lambda x.u : ()$ .
    \item[Case $t = u\,v$:] By the induction hypothesis we have $\emptyset \vdash_k u : ()$ and $\emptyset \vdash_k v : ()$.
    For $R = \emptyset$ we have $() = R(()) \Rightarrow ()$ and $\emptyset \cup R(\emptyset) = \emptyset$.
    Therefore, by rule ($\Rightarrow$E) we have $\emptyset \vdash_k u\,v : ()$.\qedhere
\end{description}
\end{proof}

Similar systems:
\begin{itemize}
\item The system $\lambda^{\text{CIL}}$ by Wells et al
  \cite{WellsDMT02} (see \cite{WellsH02} for a simplified version of
  the system) is a system in Church style with explicit type
  annotations in $\lambda$-binders where different derivations for the
  same subterm are managed by means of records. A record of types with
  labels from $I$ is derived here by a record of terms of
  corresponding labels from $I$.
\item The proof system ISL presented by Pimentel, Ronchi Della Rocca
  and Roversi \cite{PimentelRR12} features independent derivations in
  tuples led according to the same proof rules. Effectively this works
  like an intersection type system with vectors. However, the system
  does not have proof terms. A version of the proof system formulated
  in terms of a sequent calculus is presented in \cite{RoccaSSV10}.
\end{itemize}

Dissimilar systems:
\begin{itemize}
\item The system by Luigi Liquori and Simona Ronchi della Rocca
  \cite{LiquoriR07} is a system with explicit annotations at
  $\lambda$-binders. It introduces  terms in which variables in
  binders are annotated with indices. Then these indices serve as
  pointers to binders in derivations, and the binders in derivations
  contain types.
\end{itemize}

The following example demonstrates a type derivation in dimension $1$ for the term $\lambda x.x\,x$.
It also illustrates that the occurrence of sets of types in derivations does not necessarily increase dimension.

\begin{example}~\\
    We have $\emptyset \vdash_1 \lambda x.x\,x : (\{a, \{a\} \to b\} \to b)$ by the following derivation, omitting rule (Ax) and using $R := \{(1, 1)\}$:\medskip\\
\AxiomC{$\{x : (\{\{a\} \to b\})\} \vdash_1 x : R((a)) \Rightarrow (b)$}
\AxiomC{$\{x : (\{a\})\} \vdash_1 x : (a)$}
\RightLabel{\textnormal{($\Rightarrow$E)}}
\BinaryInfC{$\{ x : (\{a, \{a\} \to b\})\} \vdash_1 x\,x : (b)$}
\RightLabel{\textnormal{($\Rightarrow$I)}}
\UnaryInfC{$\emptyset \vdash_1 \lambda x.x\,x : (\{a, \{a\} \to b\} \to b)$}
\DisplayProof
\end{example}

The following example illustrates the action of relations in order to introduce intersections.

\begin{example}~\\
For the following definitions 
\begin{itemize}
\item $R_1 := \{(1,1), (2,1)\}$
\item $R_2 := \{(1,1), (1,2)\}$
\item $\Gamma_1 := \{x : (\{b, c\} \to a)\}$
\item $\Gamma_2 := \{y : (\{a\} \to b, \{a\} \to c), z : (a, a)\} = \{y : (\{a\} \to b, \{a\} \to c)\} \cup R_2(\{z : (a)\})$
\end{itemize}
we have the following derivation:\medskip\\
\AxiomC{$\Gamma_1 \vdash_2 x : R_1((b, c)) \Rightarrow (a)$}
\AxiomC{$\Gamma_2 \vdash_2 y\,z : (b, c)$}
\RightLabel{\textnormal{($\Rightarrow$E)}}
\BinaryInfC{$\Gamma_1 \cup R_1(\Gamma_2) \vdash_2 x\,(y\,z) : (a)$}
\DisplayProof\medskip\\
for which the judgment $\Gamma_2 \vdash_2 y\,z : (b, c)$ is derived in dimension $2$ as follows:\medskip\\
\AxiomC{$\{y : (\{a\} \to b, \{a\} \to c)\} \vdash_2 y : R_2((a)) \Rightarrow (b, c)$}
\AxiomC{$\{z : (a)\} \vdash_2 z : (a)$}
\RightLabel{\textnormal{($\Rightarrow$E)}}
\BinaryInfC{$\Gamma_2 \vdash_2 y\,z : (b, c)$}
\DisplayProof
\end{example}

The dimensional bound limits the length of vectors occurring in a derivation, and therefore can be weakened to any larger bound.

\begin{lemma}[Dimensional Restriction]
Any vector occurring in a derivation of $\Gamma \vdash_k t : \bar{A}$ has at most $k$ coordinates.
\end{lemma}

\begin{proof}
    Induction on $t$.
\end{proof}

\begin{lemma}[Dimension Weakening]
    \label{lemma:dimension-weakening}
    If $\Gamma\vdash_k t : \bar{A}$ and $k \leq m$ then $\Gamma\vdash_m t:\bar{A}$.    
\end{lemma}
\begin{proof}
    Induction on $t$.
    For rule (Ax) the dimension restriction $n\leq k$ is replaced by $n\leq m$, which holds given $k \leq m$.
\end{proof}

Any type assumption can be weakened (increasing the cardinality of the corresponding assigned set of types) with no effect on dimension. 

\begin{lemma}[Weakening]
\label{lem:weak}
If $\Gamma \vdash_k t : \bar{A}$ and $\Gamma\rbar{\subseteq} \Delta$, then $\Delta \vdash_k t : \bar{A}$.
\end{lemma}

\begin{proof}
    Assuming $\Gamma \vdash_k t : \bar{A}$ and $\Gamma\rbar{\subseteq} \Delta$, we proceed by induction on $t$.
    \begin{description}
    \item[Case $t = x$:] Due to $\bar{A} \rbar{\in} \Gamma(x) \rbar{\subseteq} \Delta(x)$ we obtain the claim by rule (Ax).
    \item[Case $t = \lambda x.u$:] For some $\bar{\sigma}$ and $\bar{B}$ we have $\bar{A} = \bar{\sigma} \Rightarrow \bar{B}$, and by the induction hypothesis we have $\Delta, x : \bar{\sigma} \vdash_k u : \bar{B}$.
    By rule ($\Rightarrow$I) we obtain the claim.
    \item[Case $t = u\,v$:] For some $\Delta'$, $\Gamma'$, $\bar{B}$, and $R$ such that $\Gamma = \Gamma' \cup R(\Delta')$ we have $\Gamma' \vdash_k u : R(\bar{B}) \Rightarrow \bar{A}$ and $\Delta' \vdash_k v : \bar{B}$.
    Due to $\Gamma' \rbar{\subseteq} \Gamma \rbar{\subseteq} \Delta$ by the
    induction hypothesis we have $\Delta \vdash_k u : R(\bar{B}) \Rightarrow \bar{A}$.
    By rule ($\Rightarrow$E) we have $\Delta \cup R(\Delta') \vdash_k u\,v : \bar{A}$.
    Due to $R(\Delta') \rbar{\subseteq}  \Gamma \rbar{\subseteq} \Delta$ we have 
    $\Delta \cup R(\Delta') = \Delta$, obtaining the claim.\qedhere
\end{description}
\end{proof}

A key technical insight for the parallel system is the following Lemma~\ref{lem:transform}, which allows us to permute, remove, or copy vector coordinates in derivations.

\begin{lemma}
\label{lem:transform}
If $\Gamma \vdash_k t : \bar{A}$ and $R \subseteq \{1, \ldots, \dim(\bar{A})\} \times \{1, \ldots, m\}$ is right-unique and right-total, then $R(\Gamma) \vdash_{\max(k,m)} t : R(\bar{A})\!\!\downarrow$.
\end{lemma}

\begin{proof}
    Assuming $\Gamma \vdash_k t : \bar{A}$ and that $R$ is right-unique and right-total, we proceed by induction on $t$.
    Let $n=\max(k,m)$.
    \begin{description}
    %
    \item[Case $t = x$:] We have $\bar{A}_{[i]} \in \Gamma(x)_{[i]}$ for $i \in \{1, \ldots, \dim(\bar{A})\}$.
    Therefore, we have ${(R(\bar{A})\!\!\downarrow)_{[j]}}\in R(\Gamma(x))_{[j]}$ for $j \in \{1, \ldots, \dim(R(\bar{A}))\}$ (cf.\ Lemma~\ref{lem:rel-properties}), which by rule (Ax) implies $R(\Gamma) \vdash_n x : R(\bar{A})\!\!\downarrow$.
    \item[Case $t = \lambda x.u$:] For some $\bar{\sigma}$ and $\bar{B}$ we have $\bar{A} = \bar{\sigma} \Rightarrow \bar{B}$, and by the induction hypothesis we have %
    %
    $R(\Gamma), x : R(\bar{\sigma}) \vdash_n u : R(\bar{B})\!\!\downarrow$.
    %
    As  $R$ is right-unique (cf.\ Lemma~\ref{lem:rel-properties}), $R(\bar{\sigma}) \Rightarrow R(\bar{B}) = R(\bar{\sigma} \Rightarrow \bar{B}) = R(\bar{A})$.
    By rule ($\Rightarrow$I) we obtain $R(\Gamma) \vdash_n \lambda x.u : R(\bar{A})\!\!\downarrow$.
    %
    \item[Case $t = u\,v$:] For some $\Delta$, $R'$, and $\bar{B}$ we have $\Delta \vdash_k v : \bar{B}$, and weakening the environment by Lemma~\ref{lem:weak}, we have $\Gamma \vdash_n u : R'(\bar{B}) \Rightarrow \bar{A}$.
    By the induction hypothesis we obtain $R(\Gamma) \vdash_n u : R(R'(\bar{B}) \Rightarrow \bar{A})\!\!\downarrow$.
    By Lemma~\ref{lemma:dimension-weakening} we have $\Delta \vdash_n v : \bar{B}$.
    Let $R'' = R \circ R'$. As $R$ is right-unique, we have $R(R'(\bar{B}) \Rightarrow \bar{A}) = R''(\bar{B}) \Rightarrow R(\bar{A})$ and $R(\Gamma) \cup R''(\Delta) = R(\Gamma \cup R'(\Delta))$ (cf.~Lemma~\ref{lem:rel-properties}).
    By rule ($\Rightarrow$E) we obtain $R(\Gamma \cup R'(\Delta)) \vdash_n u\,v : R(\bar{A})\!\!\downarrow$.\qedhere
\end{description}
\end{proof}

The following Example~\ref{xmp:deduplicate} illustrates use of the above Lemma~\ref{lem:transform} to permute, remove, or copy vector coordinates in derivations.

\begin{example}\label{xmp:deduplicate}
    Assume $\Gamma \vdash_k t : (A_1, A_2, A_3)$. By Lemma~\ref{lem:transform} we have:
    \begin{itemize}
        \item $R_1(\Gamma) \vdash_{\max(k, 2)} t : (A_3, A_1)$ where $R_1 := \{(1, 2), (3, 1)\}$
        \item $R_2(\Gamma) \vdash_{\max(k, 4)} t : (A_1, A_2, A_2, A_3)$ where $R_2 := \{(1, 1), (2, 2), (2, 3), (3, 4)\}$
    \end{itemize}
\end{example}

A subtle aspect of Lemma~\ref{lem:transform} is that coordinate copies do not increase the dimension besides the space required for their occurrence.

\begin{example}
    Let $\sigma := \{a,b,c\}$.
    We have $\{x : (\sigma), y : (\sigma \to d)\} \vdash_3 y\,x : (d)$, and $3$ is the minimal dimension necessary to derive this judgment.
    By Lemma~\ref{lem:transform}, using the relation $\{(1,1), (1,2)\}$ we have $\{x : (\sigma, \sigma), y : (\sigma \to d, \sigma \to d)\} \vdash_3 y\,x : (d, d)$, essentially copying the derivation.
    Still, dimension~$3$ suffices to account for the copied coordinate shown in the following derivation:\medskip\\
    \AxiomC{$\{y : (\sigma \to d, \sigma \to d)\} \vdash_3 y : R((a,b,c)) \Rightarrow (d, d)$}
    \AxiomC{$\{x : (\{a\},\{b\},\{c\})\} \vdash_3 x : (a,b,c)$}
    \RightLabel{\textnormal{($\Rightarrow$E)}}
    \BinaryInfC{$\{x : (\sigma, \sigma), y : (\sigma \to d, \sigma \to d)\} \vdash_3 y\,x : (d, d)$}
    \DisplayProof\medskip\\
    where $R := \{(1,1),(1,2),(2,1),(2,2),(3,1),(3,2)\}$ and $R((\{a\},\{b\},\{c\})) = (\sigma, \sigma)$.
\end{example}

The substitution lemma requires tight control over the \edmark

\begin{lemma}[substitution lemma]
    If $\Gamma, x : R(\bar{A}) \vdash_k t : \bar{B}$ and $\Delta \vdash_k u : \bar{A}$, then $\Gamma \cup R(\Delta) \vdash_k t[x := u] : \bar{B}$.
\end{lemma}

\begin{proof}
    Assuming $\Gamma, x : R(\bar{A}) \vdash_k t : \bar{B}$ and $\Delta \vdash_k u : \bar{A}$, we proceed by induction on $t$. We also note here that $\Gamma, x : R(\bar{A}) \vdash_k t : \bar{B}$ implies that
    $R\subseteq\{1,\ldots,\dim(\bar{B})\}\times\{1,\ldots,m\}$ where $m\leq k$.
    W.l.o.g. $\bar{A}$ has distinct coordinates as we can remove duplicate ones by Lemma~\ref{lem:transform}.
    \begin{description}
    \item[Case $t = x$:] We have $\bar{B}_{[i]} \in R(\bar{A})_{[i]}$ for $i \in \{1, \ldots, \dim(\bar{B})\}$.
    Therefore, there is a right-unique $R' \subseteq R$ such that $R'(\bar{A}) = \bar{B}$.
    By Lemma \ref{lem:transform} we have $R'(\Delta) \vdash_k u : R'(\bar{A})\!\!\downarrow$.
    Due to $R' \subseteq R$, by the weakening lemma (cf.\ Lemma~\ref{lem:weak}) we have $\Gamma \cup R(\Delta) \vdash_k u : \bar{B}$.
    \item[Case $t = y \neq x$:] We have $\Gamma \vdash_k y : \bar{B}$ and the claim follows by the weakening lemma (cf. Lemma~\ref{lem:weak}).
    \item[Case $t = \lambda y.v$ such that $y \neq x$ and $y\not\in\var(u)$:] For some $\bar{\sigma}$ and $\bar{C}$ we have $\bar{B} = \bar{\sigma} \Rightarrow \bar{C}$ and %
    %
    $\Gamma, x : R(\bar{A}), y : \bar{\sigma} \vdash_k v : \bar{C}$.
    %
    By the induction hypothesis we have $\Gamma \cup R(\Delta), y : \bar{\sigma} \vdash_k v[x := u] : \bar{C}$.
    The claim follows by rule ($\Rightarrow$I).
    \item[Case $t = v\,w$:] For some $\Gamma'$, $\Delta'$, $R'$, $\bar{\sigma}$, $\bar{\tau}$, and $\bar{C}$ such that $\Gamma = \Gamma' \cup R'(\Delta')$ and $R(\bar{A}) = \bar{\tau} \cup R'(\bar{\sigma})$ we have $\Gamma', x : \bar{\tau} \vdash_k v : R'(\bar{C}) \Rightarrow \bar{B}$ and %
    $\Delta', x : \bar{\sigma} \vdash_k w : \bar{C}$.
    %
    Note that $R'\subseteq\{1,\ldots,\dim(\bar{C})\}\times\{1,\ldots,\dim(\bar{B})\}$. 
    W.l.o.g.~$R'$ is left-total. 
    %(NOTE: $R'$ should not be be left-unique - then the dimension may increase).
    In order to obtain the claim by rule ($\Rightarrow$E) we consider the corresponding premises.
    \begin{itemize}
        \item Due to $\bar{\tau} \subseteq R(\bar{A})$, by the weakening lemma 
        (cf.\ Lemma~\ref{lem:weak}) and the induction hypothesis we have $\Gamma' \cup R(\Delta) \vdash_k v[x := u] : R'(\bar{C}) \Rightarrow \bar{B}$.
        %
        \item Due to $R'(\bar{\sigma}) \subseteq R(\bar{A})$ and since $R'$ is left-total and $\bar{A}$ has unique coordinates, by Lemma~\ref{lem:rel-properties} there exists a $R''$ such that $R'\circ R''\subseteq R$ and $R''(\bar{A}) = \bar{\sigma}$.
        By the induction hypothesis we have $\Delta' \cup R''(\Delta) \vdash_k w[x := u] : \bar{C}$
    \end{itemize}
    By rule ($\Rightarrow$E) we have $\Gamma' \cup R(\Delta) \cup R'(\Delta' \cup R''(\Delta)) \vdash_k (v\,w)[x := u] : \bar{B}$.
    By the weakening lemma (cf. Lemma~\ref{lem:weak}), it suffices to show that $\Gamma' \cup R(\Delta) \cup R'(\Delta') \cup R'(R''(\Delta)) \subseteq \Gamma \cup R(\Delta)$.
    This follows from $\Gamma = \Gamma' \cup R'(\Delta')$, $R' \circ R'' \subseteq R$. \qedhere
    %and Lemma~\ref{lem:rel-properties}. 
\end{description}
\end{proof}

\edmark: (subject reduction)

\section{Relation to Traditional Intersection Type Systems}

As a technical tool to establish relation with the traditional intersection type systems, we use the strict system of van Bakel \cite{Bakel04}, \cite[Section~5]{Bakel11}. This serves us to establish relation with the system of Barendregt, Coppo, Dezani-Ciancaglini \cite{BarendregtCD83}, which is expressed in Theorem~\ref{theorem:bcd} below.


\begin{definition}[Strict Intersection Type System~{\cite[Definition~5.1]{Bakel11}}]
\label{def:strict-type-system}
~\\
Here are the rules of the strict system slightly adapted to notation used in this paper.

~

\begin{minipage}{\textwidth}
\centering
\begin{tabular}{l}
{\RightLabel{\textnormal{($\cap$E)}}
\AxiomC{$A \in \sigma$ }
\UnaryInfC{$\Gamma, x : \sigma \vdash_S x : A$}
\DisplayProof}\quad
{\RightLabel{\textnormal{($\cap$I)}}
\AxiomC{$\Sigma\vdash_S t : A_1, \ldots, \Sigma\vdash_S t : A_n$}
\UnaryInfC{$\Sigma\vdash_S t : \{A_1,\ldots,A_n\}$}
\DisplayProof}
\\\\
{\RightLabel{\textnormal{($\to$I)}}
\AxiomC{$\Sigma, x: \sigma \vdash_S t : A$}
\UnaryInfC{$\Sigma \vdash_S \lambda x.t : \sigma \to A$}
\DisplayProof}\quad
{\RightLabel{\textnormal{($\to$E)}}
\AxiomC{$\Sigma \vdash_S t : \sigma \to B$}
\AxiomC{$\Sigma \vdash_S u : \sigma$}
\BinaryInfC{$\Sigma \vdash_S t\, u : B$}
\DisplayProof}
\end{tabular}
\end{minipage}
\end{definition}

\begin{lemma}
    \label{lemma:fromvanBakel}
    %Assume that $\Sigma_i = \{ x_1:\sigma^1_i,\ldots, x_m:\sigma^m_i\}$ for $i\in\{1,\ldots, n\}$ as %well as $\bar\sigma^j = (\sigma^j_1,\ldots,\sigma^j_n)$ and $\bar{A} = (A_1,\ldots, A_n)$. 

    If $\Gamma_{[i]}\vdash_S t:\bar{A}_{[i]}$ for each $i\in\{1,\ldots, \dim(\bar{A})\}$ then $\Gamma\vdash t:\bar{A}$.
\end{lemma}
\begin{proof}
    The proof is by induction over the structure of $t$. Assume the notation of the lemma formulation.

     \begin{description}
        %
        \item[Case $t=x$.] If $\Gamma_{[i]}\vdash_S x:\bar{A}_{[i]}$ for each $i$ then each $\Gamma_{[i]}$ contains declaration $x:\bar{\sigma}_{[i]}$ such that $\bar{A}_{[i]}\in\bar{\sigma}_{[i]}$. 
        Consequently, $\bar{A}\rbar{\in}\bar{\sigma}$. \edmark
        %
        \item[Case $t=\lambda x.u$.] If  $\Gamma_{[i]}\vdash_S \lambda x.u:\bar{A}_{[i]}$ for each $i\in\{1,\ldots,n\}$, then these judgments must be result of ($\to$I) rule. Consequently, $\bar{A}_{[i]} = \bar{\sigma}_{[i]}\to \bar{B}_{[i]}$ for some $\bar{\sigma}$ and $\bar{B}$ and we have derivations for $\Gamma_{[i]},x:\bar{\sigma}_{[i]}\vdash_S u:\bar{B}_{[i]}$. By the induction hypothesis, we obtain that $\Gamma, x:\bar{\sigma}\vdash u:\bar{B}$. We can now apply ($\Rightarrow$I) rule and obtain $\Gamma\vdash \lambda x.u:\bar{\sigma}\Rightarrow \bar{B}$.
        %
        \item[Case $t=u\; v$.] If  $\Gamma_{[i]}\vdash_S u\, v:\bar{A}_{[i]}$ then this can be only derived by ($\to$E) rule. Consequently, we have derivations for 
        $\Gamma_{[i]}\vdash_S u : \bar{\sigma}_{[i]}\to \bar{A}_{[i]}$ and $\Gamma_{[i]}\vdash_S v:\bar{\sigma}_{[i]}$.
        Then the only way to derive $\Gamma_{[i]}\vdash_S v:\bar{\sigma}_{[i]}$ is to use ($\cap$I) rule so that $\Gamma_{[i]}\vdash_S v:B^j_i$ where for some $k_i$  we have $\bar{\sigma}_{[i]} = \{ B^1_i,\ldots,B^{k_i}_i\}$. 
        Assume $\Gamma = \{ x_1 :\bar{\sigma}^1,\ldots,x_m:\bar{\sigma}^m\}$. 
        By the induction hypothesis we construct a derivation for 
        $\Delta\vdash v:\bar{B}$ where 
        \begin{displaymath}
            \begin{array}{l}
                \Delta = \{ x_1: \bar\tau^1,\ldots,x_m:\bar\tau^m\}\\
                %
                \bar\tau^i = (\underbrace{\bar{\sigma}^i_{[1]},\ldots\bar{\sigma}^i_{[1]}}_{k_1 \mbox{\small times}},
                              \ldots, %
                              \underbrace{\bar{\sigma}^i_{[n]},\ldots\bar{\sigma}^{i}_{[n]}}_{k_n \mbox{\small times}}
                ) \quad \mbox{ for } i\in\{1,\ldots,m\}\\
                % 
                \bar{B} = (B^1_1,\ldots,B^{k_1}_1, %
                 \ldots, %
                 B^1_n,\ldots,B^{k_n}_n)
            \end{array}
        \end{displaymath}
        Consider now, $l_0 =0$ and $l_i = \sum_{j=1}^{i}k_j$. We define relation 
        %
        $R = \{ (p,q) \mid q\in\{1,\ldots,n\} \mbox{ and } l_{q-1} < p \leq l_q\}$.
        %
        Observe that 
        \begin{equation}
        \label{eq:bartimes}
            \begin{array}{l}
                 R(\bar{B})_{[j]} = \bar{\sigma}_{[j]} \quad\mbox{and}\quad
                 %
                R(\Delta(x_i)) = \bar{\sigma}^i\\
            \end{array}
        \end{equation}
        for $j\in\{1,\ldots,n\}$ and $i\in\{1,\ldots,m\}$. From
        the induction hypothesis we get $\Gamma\vdash u:\bar\sigma\Rightarrow\bar{A}$, and then
        we can apply rule ($\Rightarrow$E) to obtain
        $\Gamma \cup R(\Delta)\vdash u\, v:\bar{A}$.
        Note that $R(\Delta) = \Gamma$ by observation (\ref{eq:bartimes}) above. So we indeed arrive at a derivation for $\Gamma\vdash u\, v:\bar{A}$.\qedhere
     \end{description}
\end{proof}

\begin{lemma}
    \label{lemma:toS}
    If $\Gamma\vdash t:\bar{A}$ then 
    for each $i\in\{1,\ldots,\dim(\bar{A})\}$ we have
    $\Gamma_{[i]}\vdash_S t:\bar{A}_{[i]}$.
\end{lemma}
\begin{proof}
    The proof is by induction over the structure of $t$. Assume the notation of the lemma formulation.
    \begin{description}
        \item[Case $t=x$.] If $\Gamma\vdash x:\bar{A}$ then it can be derived only by (Ax) rule. Consequently, there is $(x:\bar{\sigma}) \in\Gamma$ such that $\bar{A} \rbar{\in} \bar{\sigma}$. Consequently, we can use ($\cap$E) rule to obtain $\Gamma_{[i]}\vdash_S x:\bar{A}_{[i]}$ for $i\in\{1,\ldots, n\}$.
        %
        \item[Case $t=\lambda x.u$.] The judgment $\Gamma\vdash \lambda x.u:\bar{A}$ can be derived only by ($\Rightarrow$I) rule and thus $\bar{A}=\bar{\sigma}\Rightarrow\bar{B}$ for some $\bar{\sigma}$ and $\bar{B}$ and also $\Gamma,x:\bar{\sigma}\vdash u:\bar{B}$
        is derivable. By the induction hypothesis, we obtain that for each $i\in\{1,\ldots, n\}$ derivable is $\Gamma_{[i]},x:\bar{\sigma}_{[i]}\vdash_S u:\bar{B}_{[i]}$. Then rule ($\to$I) gives
        us derivability of $\Gamma_{[i]}\vdash_S \lambda x.u:\bar{\sigma}_{[i]}\to\bar{B}_{[i]}$. As 
        $\bar{\sigma}_{[i]}\to\bar{B}_{[i]} =\bar{A}_{[i]}$ we get the required conclusion.
        %
        \item[Case $t=u\; v$.] The judgment $\Gamma\vdash u\, v:\bar{A}$  can be derived only by ($\Rightarrow$E) rule. Consequently, $\Gamma = \Gamma'\cup R(\Delta)$ for some $\Delta$ and derivable are 
        %
        $\Gamma'\vdash u:\bar{\sigma}\Rightarrow\bar{A}$ as well as 
        %
        $\Delta\vdash v:\bar{B}$ 
        %
        where $R(\bar{B})_{[j]} = \bar{\sigma}_{[j]}$. By the induction hypothesis, we can derive
        \begin{equation}
        \label{eq:bardownstairs}
            \begin{array}{ll}
                 \Gamma'_{[i]}\vdash_S u:\bar{\sigma}_{[i]}\to\bar{A}_{[i]} 
                 & \mbox{ for } i\in\{1,\ldots,n\}\\
                 \Delta_{[j]}\vdash_S v:\bar{B}_{[j]} 
                 & \mbox{ for } j\in\{1,\ldots,\dim(\bar{B})\}
            \end{array}
        \end{equation}
        For each $i\in\{1,\ldots,n\}$ we consider indices  
        $\mathcal{I}_i = \{ p \mid (p,i)\in R\}$. %
        %
        Observe that $\Gamma'(x)\cup R(\Delta)(x)\rbar{\subseteq} \Gamma(x)$
        for $x\in\dom(\Gamma)$ and consequently $(R(\Delta)(x))_{[i]}\subseteq(\Gamma(x))_{[i]}$.
        This implies that $(\Delta(x))_{[j]}\subseteq(\Gamma(x))_{[i]}$ for $j\in\mathcal{I}_i$.
        We can now apply the weakening lemma to derivations in (\ref{eq:bardownstairs})
        and obtain %
        %
        $\Gamma_{[i]}\vdash_S v:\bar{B}_{[j]}$ for $j\in\mathcal{I}_i$ as well as
        %
        $\Gamma_{[i]}\vdash_S u:\bar\sigma_{[i]}\to\bar{A}_{[i]}$.
        %
         We use ($\cap$I) rule now to derive 
         %
         $\Gamma_{[i]}\vdash_S v:\{ \bar{B}_{[j]} \mid j\in\mathcal{I}_i\}$.
         %
         Application of ($\to$E) rule gives $\Gamma_{[i]}\vdash_S u\, v:\bar{A}_{[i]}$ for each $i\in\{1,\ldots,n\}$. \qedhere
    \end{description}
\end{proof}

As a final step, we establish the connection with the system BCD of Barendregt, Coppo, Dezani-Ciancaglini \cite{BarendregtCD83}.
The set of types in this system is bigger than the set of strict types of van Bakel.
However, for each type $\varphi$ in this system there is a set of strict types $A_1,\ldots,A_n$ such that $\varphi$ is
equivalent to their intersection (cf.\ \cite[Lemma 2.2.2]{Bakel92}).
This equivalence is the canonical equivalence induced by the quasi-order $\leq$ that is the subtyping relation in BCD system.
Therefore, we do not lose any generality by writing $\Sigma\vdash_{BCD} t:\sigma$, where $\Sigma$ and $\sigma$ comply to the syntax used in this paper.
We can also write $\Sigma\vdash_{BCD} t:A$ as strict types are a special case of BCD types.
%
%We can embed the intersection types from Definition~\ref{def:itypes} using the following
%translation:
%\begin{itemize}
%    \item $a^\triangleleft = a$
%    \item $(\sigma\to A)^\triangleleft = \sigma^\triangleleft\to A^\triangleleft$
%    \item $\emptyset^\triangleleft = \omega$
%    \item $\{A_1,\ldots,A_n\}^\triangleleft = 
%        A_1^\triangleleft\land\cdots\land A_n^\triangleleft$. 
%\end{itemize}
%The translation naturally extends to environments by application of the transformation above to types.
%
%The formulation of the theorem that relates the strict system with the system of BCD
%uses the standard semantical ordering $\leq$ on types. We recall it here for completeness. We say that $\varphi\leq\psi$ when the relation can be established by means of the following axioms:
%\begin{itemize}
%    \item $\varphi\leq\varphi$,
%    \item $\varphi\leq\omega$ 
%    \item $\omega\leq\omega\to\omega$ 
%    \item $\varphi\land\psi\leq\varphi$
%    \item $\varphi\land\psi\leq\psi$
%    \item $(\varphi\to\psi)\land(\varphi\to\psi')\leq \varphi\to(\psi\land\psi')$
%    \item $\varphi\leq\varphi'\leq\varphi'' \Rightarrow \varphi\leq\varphi''$
%    \item $\varphi\leq\psi$ and $\varphi\leq\psi' \Rightarrow \varphi\leq(\psi\land\psi')$
%    \item $\varphi'\leq\varphi$ and $\psi\leq\psi' \Rightarrow \varphi\to\psi\leq\varphi'\to\psi'$
%\end{itemize}
 To establish the relation between our system and the system BCD, we use the following statement mentioned in \cite[Section~5]{Bakel11}. 
\begin{theorem}[van Bakel, \cite{Bakel11}]
    \label{theorem:vanBakelS}
    If $\Sigma\vdash_{BCD} t:\sigma$ then there are $\Sigma',\{A_1,\ldots,A_n\}$ such that $\Sigma\leq\Sigma'$, $\Sigma' \vdash_{S} t:A_i$ for each $i\in\{1,\ldots,n\}$ and $\{A_1,\ldots,A_n\}\leq\sigma$.

    Moreover, If $\Sigma \vdash_{S} t:A$ then $\Sigma\vdash_{BCD} t:A$.
\end{theorem}
A similar theorem was proved in \cite{Bakel92}  (Theorem 2.2.6) for another variant of the strict system the rules of which are derivable in the strict system of van Bakel used in this paper.

Note that the formulation above omits the condition that $t$ is in $\lambda\bot$-normal form, which is present in \cite{Bakel92}.
This condition is trivially fulfilled as our terms do not use the constant $\bot$.
In addition, the original formulation in \cite{Bakel92} does not include the second part, which obtains a~derivation in BCD from the one in the strict system.
However, this part easily follows by observation that the rules of the strict system in its presentation in \cite{Bakel11} are derivable in BCD system.
We need one more result concerning the type ordering.
\begin{proposition}
    \label{prop:picking}
    If $\sigma\leq\sigma'$ then
    there is $\tau\subseteq\sigma$ such that $|\tau|\leq  |\sigma'|$ and $\tau\leq\sigma'$.
\end{proposition}
\begin{proof}
    For the proof we use an observation made by Hindley, cf. \cite[Lemma~3(i)]{Hindley82}, which also follows by easy induction from the work of Barendregt, Coppo, Dezani-Ciancaglini \cite[Lemma 2.4(ii)]{BarendregtCD83}.
    \begin{quote}
        If $\{A_1,\ldots,A_n\}\leq \{B_1,\ldots,B_m\}$ then for each $i\in\{1,\ldots,m\}$ there is $j\in\{1,\ldots,n\}$ such that $A_j\leq B_i$.
    \end{quote}
    To prove the current proposition we observe that $\sigma$ can be presented as $\{A_1,\ldots,A_n\}$ and $\sigma'$ as $\{B_1,\ldots,B_m\}$. By the observation above, we take for each $i\in\{1,\ldots,m\}$
    only one $A_{j_i}$ such that $A_{j_i}\leq B_i$. We obtain then immediately that $\{A_{j_1},\ldots,A_{j_m}\}\leq\sigma'$.
\end{proof}

The considerations above give us the following comparison of derivability in our system and the system BCD.
\begin{theorem} ~
    \label{theorem:bcd}
    \begin{enumerate}
        \item If $\Gamma \vdash_k t : \bar{A}$ then for $i\in\{1,\ldots,\dim(\bar{A})\}$
             we have $\Gamma_{[i]}\vdash_{\textnormal{BCD}} t : \bar{A}_{[i]}$.

        \item If $\Sigma \vdash_{\textnormal{BCD}} t : \sigma$ where $\sigma = \{A_1,\ldots, A_n\}$
            then there are $\Gamma,\bar{B}$ with $\dim(\Gamma)=\dim(\bar{B})=n$
            such that $\Sigma \leq\Gamma_{[i]}$, $\Gamma \vdash_k t : \bar{B}$ for some $k$ and
                $\{\bar{B}_{[1]},\ldots,\bar{B}_{[n]}\}\leq \sigma$.
    \end{enumerate}
\end{theorem}
\begin{proof}
    For the proof of (1), we assume $\Gamma \vdash_k t : \bar{A}$, which by Lemma~\ref{lemma:toS} gives us that $\Gamma_{[i]}\vdash_S t : \bar{A}_{[i]}$ for $i\in\{1,\ldots,\dim(\bar{A})\}$.
    Then the second statement of Theorem~\ref{theorem:vanBakelS} gives us that $\Gamma_{[i]}\vdash_{BCD} t:\bar{A}_{[i]}$.

    For the proof of (2), we assume that $\Sigma \vdash_{\textnormal{BCD}} t : \sigma$. By the first statement of Theorem~\ref{theorem:vanBakelS}, we obtain an environment $\Sigma'$ such that $\Sigma\leq\Sigma'$ and a type $\sigma' = \{B_1,\ldots, B_{m}\}$ such that
    $\Sigma' \vdash_S t : B_i$ for all $i\in\{1,\ldots,m\}$ and %
    $\{B_1,\ldots, B_{m}\}\leq \sigma$ for some $m$ (*). 
    Note that we may assume that $m=n$.
    Indeed, when $n<m$ we just introduce more copies of some coordinate $i$, cf.\ Lemma~\ref{lem:transform}.
    When $m<n$, we use Proposition~\ref{prop:picking} to obtain some $\tau\subseteq\sigma$ the size of which
    is not greater than the size of $\sigma'$ and by the inclusion $\tau\subseteq\sigma$ we still
    have $\Sigma' \vdash_S t : B$  for all $B\in\tau$.
    %
    This means that $\Sigma \vdash_{S} t : B_i$ for each  $i\in\{1,\ldots,n\}$.
    Consider now $\bar{B} = (B_1,\ldots,B_n)$. 
    We now define a context $\Gamma$ so that $\dom(\Gamma)=\dom(\Sigma)$ and $(x:\bar{\rho}_x)\in\Gamma$ for
    $\bar{\rho}_x = (\underbrace{\rho,\ldots,\rho}_{n\mbox{\scriptsize -times}})$
    where $(x:\rho)\in\Sigma$. Note that $\Gamma_{[i]} = \Sigma$ so Lemma~\ref{lemma:fromvanBakel} gives us that  $\Gamma\vdash_k t:\bar{B}$ for some $k$. 
    Note that $\{\bar{B}_{[1]},\ldots,\bar{B}_{[n]}\}\leq\sigma$ by (*).
\end{proof}


\newpage

\section{Decidability of Inhabitation in Bounded Dimension}

In this section, we give the main result of the present work: inhabitation in bounded dimension is decidable in \textsf{2-EXPTIME}.
This result is supplemented by the observation that for each fixed dimension inhabitation is decidable in \textsf{EXPTIME}.

\begin{definition}[Inhabitation in Bounded Dimension]
    Given $\Gamma$, $\bar{A}$, and $k$ is the judgment $\Gamma \vdash_k t : \bar{A}$ derivable for some $t$?
\end{definition}

The proof of decidability is constructed in three steps.
First, we establish inversion properties (Lemma~\ref{lem:lam_inversion} and Lemma~\ref{lem:apps_inversion}) for abstraction and iterated application.
Second, we formulate a subformula property (Lemma~\ref{lemma:subformula}) for $\beta$-normal form derivations to restrict the inhabitant search space.
Third, we give an alternating exponential space inhabitant search algorithm, obtaining a \textsf{2-EXPTIME} upper bound on inhabitation (Theorem~\ref{thm:upper_complexity_bound}).

\begin{lemma}\label{lem:lam_inversion}
    If $\Gamma \vdash_k \lambda x.t : \bar{B}$, then $\bar{B} = \bar{\sigma} \Rightarrow \bar{A}$ for some $\bar{\sigma}$ and $\bar{A}$, and $\Gamma, x : \bar{\sigma} \vdash_k t : \bar{A}$.
\end{lemma}

\begin{proof}
    Case analysis regarding the last rule application.
\end{proof}

\begin{lemma}\label{lem:apps_inversion}
    If $\Gamma \vdash_k x\,u_1 \ldots u_n : \bar{B}$, then there exist $\bar{A}_1, \ldots, \bar{A}_n$, $R_1, \ldots, R_n$, $\Delta_1, \ldots, \Delta_n$ such that
    \begin{enumerate}
        \item\label{prop:apps_inversion:dims} $R_i \subseteq \{1, \ldots, \dim(\bar{A}_i)\} \times \{1, \ldots, k\}$ is left-total and $\dim(\bar{A}_i) \leq k$ for $i \in \{1, \ldots, n\}$
        \item\label{prop:apps_inversion:type} $R_1(\bar{A}_1) \Rightarrow \cdots \Rightarrow R_n(\bar{A}_n) \Rightarrow \bar{B} \rbar{\in} \Gamma(x)$
        \item\label{prop:apps_inversion:subset} $R_i(\Delta_i(y)) \rbar{\subseteq} \Gamma(y)$ for $i \in \{1, \ldots, n\}$ and $y \in \dom(\Delta_i)$
        \item\label{prop:apps_inversion:Delta} $\Delta_i \vdash_k u_i : \bar{A}_i$ for $i \in \{1, \ldots, n\}$
        \item\label{prop:apps_inversion:partial_type} $\Gamma \vdash_k x\,u_1\ldots u_i : R_{i+1}(\bar{A}_{i+1}) \Rightarrow \cdots \Rightarrow R_n(\bar{A}_n) \Rightarrow \bar{B}$ for $i \in \{1, \ldots, n\}$
    \end{enumerate}
\end{lemma}

\begin{proof}
    Induction on $n$ and case analysis regarding the last rule application.
    Left-totality of relations $R_1, \ldots, R_n$ for Property~(\ref{prop:apps_inversion:dims}) is established using Lemma~\ref{lem:transform}.
\end{proof}

With each type or a set of types we associate a set of subformulae (types or sets of types), by the following Definition~\ref{def:subformula}.

\begin{definition}[Subformula]\label{def:subformula}~
\begin{itemize}
    \item $\form(a) := \{a\}$
    \item $\form(\sigma \to A) := \{\sigma \to A\} \cup \form(\sigma) \cup \form(A)$
    \item $\form(\{A_1, \ldots, A_n\}) := \{\{A_1, \ldots, A_n\}\} \cup \bigcup_{i=1}^n \form(A_i)$
\end{itemize}
The notion of occurring subformulae is extended to vectors:
\begin{itemize}
    \item $\form((A_1, \ldots, A_n)) := \form(A_1) \cup \cdots \cup \form(A_n)$
    \item $\form((\sigma_1, \ldots, \sigma_n)) := \form(\sigma_1) \cup \cdots \cup \form(\sigma_n)$
\end{itemize}
\end{definition}

The number of subformulae of a given type or set of types is polynomial in the number of nodes in its syntax tree, illustrated by the following example.

\begin{example}\label{xmp:subformula_no_blowup}
    $\form(\{a, b\} \to c) = \{\{a, b\} \to c, \{a, b\}, a, b, c\}$.
    In particular, the set $\{a, b\}$ as well as the types $a, b$ are subformulae of $\{a, b\} \to c$, but the sets $\{a\}$ and $\{b\}$ are not.
\end{example}

\begin{lemma}[Subformula Property]
    \label{lemma:subformula}
    Assume that $\Gamma\vdash_k t : \bar{A}$ such that $t$ is in $\beta$-normal form.
    Let $\mathcal{F} := \form(\bar{A}) \cup \bigcup \{\form(\bar{\tau}) \mid (y : \bar{\tau}) \in \Gamma\}$ be the set of all subformulae of types and sets of types occurring in $\Gamma$ and $\bar{A}$.
    
    There is a derivation of $\Gamma\vdash_k t : \bar{A}$
    such that for each judgment $\Delta\vdash_k u : \bar{B}$ in this derivation we have
    \begin{enumerate}
        \item\label{lemma:subformula:env} for $(x:\bar{\sigma})\in\Delta$ and $i\in\{1,\ldots,\dim(\Delta)\}$ we have $\bar{\sigma}_{[i]} \in \mathcal{F} \cup \{\emptyset\}$
        \item\label{lemma:subformula:typ} $\bar{B}_{[i]} \in \mathcal{F}$ for $i\in\{1,\ldots,\dim(\bar{B})\}$
    \end{enumerate}
\end{lemma}

\begin{proof}
    Induction on the size of $t$. There are two cases.
    \begin{description}
        \item[Case $t = \lambda x.u$:] Follows immediately from Lemma~\ref{lem:lam_inversion} and the induction hypothesis.
        \item[Case $t = x\,u_1 \ldots u_n$:] There exist $\bar{A}_1, \ldots, \bar{A}_n$, $R_1, \ldots, R_n$, $\Delta_1 \ldots \Delta_n$ satisfying properties of Lemma~\ref{lem:apps_inversion}(\ref{prop:apps_inversion:dims}) -- Lemma~\ref{lem:apps_inversion}(\ref{prop:apps_inversion:partial_type}).
        By Lemma~\ref{lem:apps_inversion}(\ref{prop:apps_inversion:partial_type}), judgments $\Gamma \vdash_k x\,u_1\ldots u_i : R_{i+1}(\bar{A}_{i+1}) \Rightarrow \cdots \Rightarrow R_n(\bar{A}_n) \Rightarrow \bar{B}$ satisfy properties (\ref{lemma:subformula:env}) and (\ref{lemma:subformula:typ}).
        While subderivations $\Delta_i \vdash_k u_i : \bar{A}_i$ for $i \in \{1, \ldots, n\}$ satisfy Property~(\ref{lemma:subformula:typ}), they do not necessarily satisfy Property~(\ref{lemma:subformula:env}).
        By Lemma~\ref{lem:apps_inversion}(\ref{prop:apps_inversion:subset}) type assumptions in $\Delta_i$ for $i \in \{1, \ldots, n\}$ may be strict subsets of subformulae in $\mathcal{F}$.
        By Lemma~\ref{lem:weak} and Lemma~\ref{lem:rel-properties}(\ref{prop:rel-properties:saturate}) we can weaken the type assumptions by saturating the sets of types to subformulae in $\mathcal{F}$.
        This results in derivation environments $\Delta_1', \ldots, \Delta_n'$ and subderivations of judgments $\Delta_i' \vdash_k u_i : \bar{A}_i$ for $i \in \{1, \ldots, n\}$ satisfying properties (\ref{lemma:subformula:env}) and (\ref{lemma:subformula:typ}).
        We obtain the claim by the induction hypothesis.
    \end{description}
\end{proof}



\begin{theorem}\label{thm:upper_complexity_bound}
Given $\Gamma$, $\bar{A}$, and $k$ it is decidable in \textsf{2-EXPTIME} whether $\Gamma \vdash_k t : \bar{A}$ holds for some $t$.
\end{theorem}

\begin{proof}
In order to decide whether $\Gamma \vdash_k t : \bar{A}$ holds for some $t$, by subject reduction and strong normalization properties (TODO ref) it suffices to consider $t$ in $\beta$-normal form.
We discuss an \textsf{AEXPSPACE} decision procedure, and obtain the result by the equivalence between \textsf{AEXPSPACE} and \textsf{2-EXPTIME}.

Let $\mathcal{F} := \form(\bar{A}) \cup \bigcup \{\form(\bar{\tau}) \mid (y : \bar{\tau}) \in \Gamma\}$ be the set of all subformulae of types and sets of types occurring in $\Gamma$ and $\bar{A}$, and let $N := |\mathcal{F}|$ denote its cardinality.
By Lemma~\ref{lemma:subformula}, we can consider a derivation of $\Gamma \vdash_k t : \bar{A}$ such that for each judgment $\Delta \vdash_k u : \bar{B}$ in this derivation each coordinate of occurring type vectors can take at most $N + 1$ distinct values (including the empty set for sets of types in environments).
Therefore, the number of distinct $k$-dimensional type vectors to consider is bounded by ${(N+1)}^{k}$.

The decision procedure works as follows.
First, we non-deterministically guess whether $t$ is of shape $\lambda x.u$ for some $x$ and $u$, or of shape $x\,u_1 \ldots u_n$ for some $x \in \dom(\Gamma)$ and some $u_1, \ldots, u_n$ where $n$ is bounded by the maximal nesting of the arrow type constructor to the right in types occurring in $\Gamma(x)$.

If $t$ is of shape $\lambda x.u$, then by Lemma~\ref{lem:lam_inversion} we have $\bar{A} = \bar{\sigma} \Rightarrow \bar{A'}$ for some $\bar{\sigma}$ and $\bar{A'}$, and we recursively decide whether $\Gamma, x : \bar{\sigma} \vdash_k u : \bar{A'}$ holds for some $u$.

If $t$ is of shape $x\,u_1 \ldots u_n$, then we use Lemma~\ref{lem:apps_inversion}, universally branching in $i \in \{1, \ldots, n\}$, and non-deterministically guessing the necessary relation $R_i$ (bounded in cardinality by $k^2$), type vector $\bar{A}_i$, and environment $\Delta_i$ to recursively decide whether $\Delta_i \vdash_k u_i : \bar{A}_i$ holds for some $u_i$.

By the pigeonhole principle, if a type environment contains more than ${(N+1)}^{k}$ variables, some must have identical type vectors, and redundant copies can be removed without affecting derivability.
In the non-deterministic guess of the shape $\lambda x.u$, if $x$ is assigned the vector $\Gamma(y)$ for some $y \in \dom(\Gamma)$, we can equivalently guess $\lambda x.u[x := y]$ instead and remove the assumption on $x$.

Overall, the space required by the alternating decision procedure is polynomial in ${(N+1)}^{k}$, yielding an \textsf{AEXPSPACE}, or equivalently, a \textsf{2-EXPTIME} decision procedure.
\end{proof}

If we fix a dimension $k$, we obtain an \textsf{EXPTIME} inhabitation decision procedure.
It is essential for the subformula property to not include strict subsets or sets of types occurring in the input (cf.~Example~\ref{xmp:subformula_no_blowup}).
Otherwise, the number of subformulae would not be polynomial in the size of the input.

\begin{theorem}
    For any $k \in \mathbb{N}$, given $\Gamma$ and $\bar{A}$ it is decidable in \textsf{EXPTIME} whether $\Gamma \vdash_k t : \bar{A}$ holds for some $t$.
\end{theorem}

\begin{proof}
    Similar to the proof of Theorem~\ref{thm:upper_complexity_bound}, the number of distinct vectors in recursive invocation of the decision procedure is ${(N+1)}^{k}$, which is polynomial for any fixed $k$.
    This yields an \textsf{APSPACE}, or equivalently, an \textsf{EXPTIME} decision procedure.
\end{proof}

\newpage

\section{Comparison with the Multiset-dimensional System}

Comparing the presented dimensionally bounded system to the currently most expressive dimensionally bounded system $(\Vdash_k)$ with decidable inhabitation~\cite[Section~3.2]{DudenhefnerR17}, one can observe several differences.

The system $(\Vdash_k)$ introduces non-idempotent intersection types\footnote{The use of non-idempotent intersection types is to measure intersection introduction as a resource, not assumption reuse as in traditional literature (TODO cite).} (having multisets to the left of the arrow) as annotations throughout the derivation.
This induces additional bookkeeping (see side condition $(\star)$ in rule $(\cap\textnormal{I})$), and the  correspondence to established intersection type systems (both idempotent and non-idempotent) is not immediate.

The most important difference is that the individual dimensions in system $(\Vdash_k)$ are strictly less comprehensive wrt. inhabited terms.
First, any derivation underlying $\Vdash_k$ can be translated to a derivation in $\vdash_k$.

\begin{lemma}\label{lem:multiset_to_vector}
    If $\Gamma_{[1]} \Vdash_k t : A$, then $\Gamma \vdash_k t : (A)$.
\end{lemma}

\begin{proof}
    Both $\Vdash_k$ and $\vdash_k$ follow derivations $\vdash_S$.
    Using lexicographic ordering, multisets of cardinality at most $k$ at top-level can be captured by vectors of size at most $k$.
    Therefore, the same construction as the proof of Lemma~\ref{lemma:fromvanBakel} applies for the translation.    
\end{proof}

Since rank $2$ normal form typings are captured by linear multiset-dimension~\cite[Proposition~23]{DudenhefnerR17}, we obtain a similar characterization as a corollary.

\begin{definition}[Rank]~\\
    $\begin{array}{rcl}
        \rank(A) &::=& 0 \text{ if } A \text{ is a simple type}\\
        \rank(\sigma \to B) &::=& \max\{\rank(\sigma) + 1, \rank(B)\} \text{ otherwise}\\
        \rank(\{A_1, \ldots, A_n\}) &::=& \max\{1, \rank(A_1), \ldots, \rank(A_n)\}
        %\rank(\Sigma) &::=& \max\{ \max\{\rank(A_1) + 1, \ldots, \rank(A_n) + 1\} \mid (x : \{A_1, \ldots, A_n\}) \in \Sigma \}
    \end{array}$
\end{definition}

\begin{corollary}[{cf.~\cite[Proposition~23]{DudenhefnerR17}}]
    \label{cor:rank2}
    Let $t$ be a term in $\beta$-normal form, let $\bar{A}$ be a vector of length $n$ such that $\rank(\bar{A}_{[i]}) = 2$ for $i \in \{1, \ldots, n\}$, and let $\Gamma$ be an environment such that $\Gamma_{[i]}(x)$ is an intersection of simple types for $i \in \{1, \ldots, n\}$ and $x \in \dom(\Gamma)$.
    If $\Gamma_{[i]} \vdash_S t : \bar{A}_{[i]}$, then $\Gamma \vdash_n : \bar{A}$.
\end{corollary}

Second, the following Lemma~\ref{lem:exp_blowup_1} shows that there are types $A_1, A_2, \ldots$ which are inhabited in dimension $2$, but each $A_i$ requires at least multiset dimension $2^i$ to be assigned to any term. 

\begin{lemma}\label{lem:exp_blowup_1}
    There exists a family of types $\{A_1, A_2, \ldots\}$ such that for all $n \in \mathbb{N}$:
    \begin{enumerate}
        \item The size (number of nodes in the syntax tree) of $A_n$ is linear in $n$.
        \item There exists a term $t_n$ such that $\emptyset \vdash_2 t_n : (A_n)$.
        \item If there exists a term $u_n$ such that $\emptyset \Vdash_k u_n : A_n$, then $k \geq 2^n$.
    \end{enumerate}
\end{lemma}

\begin{proof}
    Let $\sigma_n := \bigcup\limits_{i=1}^n \{ \{a_i, b_i\} \to a_{i+1}, \{a_i, b_i\} \to b_{i+1} \}$,
    $\tau_n := \{\{a_{n+1}, b_{n+1}\} \to c\}$
    and $A_n := \{a_0, b_0\} \to \sigma_n \to \tau_n \to c$.
    \begin{enumerate}
        \item This property holds by definition.
        \item Let $v_0 := x$ and $v_{i+1} := y\,v_i$ for $i \in \{0, \ldots, n\}$.
        Let $t_n := \lambda xyz.z\,v_n$.
        We have $\emptyset \vdash_2 t_n : (A_n)$.
        The crucial step is
            \begin{displaymath}
       {\RightLabel{\textnormal{($\Rightarrow$E)}}
         \AxiomC{$\Gamma \vdash_2 y : R((a_i, b_i)) \Rightarrow (a_{i+1}, b_{i+1})$}
        \AxiomC{$\Gamma \vdash_2 v_i : (a_i, b_i)$}
        \BinaryInfC{$\Gamma \vdash_2 y\,v_i : (a_{i+1}, b_{i+1})$}
        \DisplayProof
  }
    \end{displaymath}
    Where $\Gamma = \{x: (\{a_0, b_0\}, \{a_0, b_0\}), y: (\sigma_n, \sigma_n), z: (\tau_n, \tau_n) \}$
    and $R := \{(1,1), (1,2), (2, 1), (2,2)\}$, observing that $R(\Gamma) = \Gamma$.
    \item An inhabitant of $A_n$ is necessarily of shape $\lambda xyz.z\,(y \ldots (y\,x) \ldots)$ such that $y$ occurs at least $n$ times and each occurrence of $y$ via the $(\cap\textnormal{I})$ rule doubles the annotation multiset cardinality.\qedhere
    \end{enumerate}
    
\end{proof}

Third, complementarily to the above Lemma~\ref{lem:exp_blowup_1}, there is a family terms which are typed at a specific type $A$ in dimension $2$, but require an exponentially large multiset dimension.

\begin{lemma}\label{lem:exp_blowup_2}
    There exists a type $A$ and a family of terms $\{t_1, t_2, \ldots\}$ such that for all $n \in \mathbb{N}$:
    \begin{enumerate}
        \item The size (number of nodes in the syntax tree) of $t_n$ is linear in $n$.
        \item $\emptyset \vdash_2 t_n : (A)$.
        \item If $\emptyset \Vdash_k t_n : A$, then $k \geq 2^n$.
    \end{enumerate}
\end{lemma}

\begin{proof}
    Let $u_0 := x$ and $u_{n+1} := y\,u_n$ for $n \in \mathbb{N}$.
    Let $t_n := \lambda xyz.z\,u_n$.
    Let $\sigma := \{\{a, b\} \to a, \{a, b\} \to b\}$.
    Let $\tau := \{\{a, b\} \to c\}$.
    Let $A := \{a, b\} \to \sigma \to \tau \to c$.
    \begin{enumerate}
        \item This property holds by definition.
        \item We have $\emptyset \vdash_2 t_n : (A)$.
        The crucial step for $i \in \mathbb{N}$ is
            {\begin{displaymath}
            \RightLabel{\textnormal{($\Rightarrow$E)}}
            \AxiomC{$\Gamma \vdash_2 y : R((a, b)) \Rightarrow (a, b)$}
            \AxiomC{$\Gamma \vdash_2 u_i : (a, b)$}
            \BinaryInfC{$\Gamma \vdash_2 y\,u_i : (a, b)$}
            \DisplayProof
            \end{displaymath}}%
        Where $\Gamma = \{x: (\{a, b\}, \{a, b\}), y: (\sigma, \sigma), z: (\tau, \tau) \}$
    and $R := \{(1,1), (1,2), (2, 1), (2,2)\}$, observing that $R(\Gamma) = \Gamma$.
        \item Similarly to the proof of Lemma~\ref{lem:exp_blowup_1}, each occurrence of $y$ via the $(\cap\textnormal{I})$ rule doubles the annotation multiset cardinality.\qedhere
    \end{enumerate}
\end{proof}

Similarly to multiset-dimension the present notion of dimension is not invariant under $\eta$-expansion:
$\emptyset \vdash_1 \lambda x.x : (\{\{a, b\} \to c\} \to \{a, b\} \to c)$, but $\emptyset \not\vdash_1 \lambda x y.x\,y : (\{\{a, b\} \to c\} \to \{a, b\} \to c)$.

The avid reader may wonder: why is the \textsf{EXPSPACE} inhabitation procedure~$\mathcal{A}_{\langle d \rangle}$, which respects a bound $d$ on the multiset dimension, not suited for inhabitation in the system $(\vdash_k)$?
One immediate answer is given by the construction in the proof of Lemma~\ref{lem:exp_blowup_1}.
Iterative intersection introduction strictly increases multiset dimension.
This is reflected in procedure~$\mathcal{A}_{\langle d \rangle}$ via \emph{multisets of simultaneous constraints} with strictly increasing cardinalities on intersection introduction.

Still, the immediate answer is not satisfactory.
The construction in the proof of Lemma~\ref{lem:exp_blowup_1} can easily be accommodated for by identifying identical constraints.
The modified inhabitation procedure would remain in \textsf{EXPSPACE}.

A key property of procedure~$\mathcal{A}_{\langle d \rangle}$ is that environments in simultaneous constraints are monotone wrt.~pointwise inclusion.
That is, there is no need to strengthen type assumptions.
This is different for the system $(\vdash_k)$ for which in rule ($\Rightarrow$E) type assumptions of premises are combined, including the action of a relation $R$.
Without the monotonicity condition, the \textsf{EXPSPACE} (or, alternating exponential time) argument is not applicable.

Overall, the present notion of dimension resides between the multi-set dimension and set-dimension, while preserving decidable inhabitation.

\newpage

The following example shows more of the data flow management features of the present system.
It is more than a mere vector presentation of the strict intersection type system.

\begin{example}[Independent, Sequential Intersection Introduction]
Fix the following:
\begin{itemize}
    \item $\sigma_1 := \{f, g\}$
    \item $A_1 := \{b_1 \to e \to c\}$
    \item $A_2 := \{b_2 \to e \to d\}$
    \item $\sigma_2 := \{\{b_1\} \to \{A_1\} \to c, \{b_2\} \to \{A_2\} \to d\}$
    \item $\tau := \sigma_1 \to \{\sigma_1 \to e\} \to \{\sigma_2 \to a\} \to a$
    \item $N := \lambda rs.s\,r\,(y\,x)$
    \item $M := \lambda xyz.z\,N$
    \item $R := \{(1, 1), (1, 2)\}$
\end{itemize}
   A derivation of $\emptyset \vdash_S M : \tau$ in the strict intersection type system contains three occurrences of the intersection introduction rule ($\cap$I) with the following concluding judgments:
\begin{enumerate}
    \item $\{x : \sigma_1, y : \{\sigma_1 \to e\}, z : \{\sigma_2 \to a\}\} \vdash_S N : \sigma_3$
    \item $\{x : \sigma_1, y : \{\sigma_1 \to e\}, z : \{\sigma_2 \to a\}, r : \{b_1\}, s : \{A_1\}\} \vdash_S x : \sigma_1$
    \item $\{x : \sigma_1, y : \{\sigma_1 \to e\}, z : \{\sigma_2 \to a\}, r : \{b_2\}, s : \{A_2\}\} \vdash_S x : \sigma_1$
\end{enumerate}
    Therefore, one may assume that dimension $4$ is necessary for the consecutive intersection introductions.
    However, we have $\emptyset \vdash_2 M : \tau$.
    In the system $(\vdash_k)$, there is no reason to keep type assumptions for variables $r$ and $s$ in order to assign a type to the subterm $(y\,x)$.
    Without the irrelevant assumptions, the distinct environments are unified.
    The relation $R$ keeps track of this unification, and we can derive the following two judgments:
\begin{enumerate}
    \item[(i)] $\{r : (b_1, b_2), s : (A_1, A_2)\} \vdash_2 s\,r : R((e)) \Rightarrow (c, d)$
    \item[(ii)] $\{x : (\sigma_1), y : (\{\sigma_1 \to e\})\} \vdash_2 y\,x : (e)$
\end{enumerate}
    Using the rule ($\Rightarrow$E) we obtain 
    {\begin{displaymath}
    \RightLabel{\textnormal{($\Rightarrow$E)}}
    \AxiomC{$(i)$}
    \AxiomC{$(ii)$}
    \BinaryInfC{$\{r : (b_1, b_2), s : (A_1, A_2)\} \cup R(\{x : (\sigma_1), y : (\sigma_2)\}) \vdash_2 s\,r\,(y\,x) : (c, d)$}
    \DisplayProof
    \end{displaymath}}%
    From the above derivation the remainder of the derivation of $\emptyset \vdash_2 M : \tau$ is straightforward.
\end{example}

\end{document}
